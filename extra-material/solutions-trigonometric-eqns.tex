\documentclass{amsart}

\usepackage[a4paper]{geometry}
\usepackage{parskip}

\title{Solutions for trigonometric equations, slide deck 1d}

\begin{document}
\maketitle

Below, you can find the solutions for the 7 trigonometric problems on slides
15-21 of slide deck 1d (Trigonometric Equations), as well as a hint on how to
get started.

Note: in some cases, you may obtain a solution that \emph{looks} different from the one given here, but where both solutions are actually equivalent. For example, the following three solutions are all the same: (a) $\theta = \pi/2 + k\pi$, (b) $\theta = - \pi/2 + k \pi$, and (c) $\theta = (2k + 1) \pi/2$.

\section*{Problem 1}

\begin{itemize}
  \item Problem statement: $\tan x \tan 4x = 1$.
  \item Solution: $x = \frac{\pi}{10} + k \frac{\pi}{5}$.
  \item Hint:
    \begin{itemize}
    \item Divide both sides by $\tan 4x$ and write $\cot 4x$ as $\tan \left( \pi/2 - 4x\right)$. This gives you an equation $\tan 4x = \tan (\pi/2 - 4x)$ that you can solve with Case III.
    \item This problem was covered in class; please refer to your notes for a full solution.
    \end{itemize}
  \end{itemize}  

\section*{Problem 2}

\begin{itemize}
  \item Problem statement: $1 + \sin \theta = 2\cos^2\theta$.
  \item Solution: $\theta = (-1)^k (-\pi/2) + k \pi$, $k \in \mathbb{Z}$, or $\theta = (-1)^k \pi/6 + k \pi$, $k \in \mathbb{Z}$.
  \item Hint: The problem statement looks like a quadratic equation: write $\cos^2 \theta = 1 - \sin^2 \theta$ and substitute $t = \sin \theta$ to make it into a quadratic equation.
  \end{itemize}  

\section*{Problem 3}

\begin{itemize}
  \item Problem statement: $\sin 2\theta = \cos \theta$.
  \item Solution: $\theta = \pi/2 + k \pi$, $k \in \mathbb{Z}$, or $\theta = (-1)^k \pi/6 + k \pi$, $k \in \mathbb{Z}$.
  \item Hint: Write $\sin 2\theta = 2\sin\theta \cos\theta$ and factor out $\cos\theta$.
  \end{itemize}  

  \section*{Problem 4}

\begin{itemize}
  \item Problem statement: $\tan x - \cot x = \csc x$.
  \item Solution: $x = (2k + 1)\pi$, $k \in \mathbb{Z}$, or $x = \pm \pi/3 + 2k\pi$, $k \in \mathbb{Z}$.
  \item Hint: Expand all three trigonometric functions in terms of $\sin x$ and $\cos x$.
  \end{itemize}  

  \section*{Problem 5}

\begin{itemize}
  \item Problem statement: $\sin x + 2 \cos x = 1$.
  \item Solution: $x = 2\alpha - \pi/2 + 2\pi k$, $k \in \mathbb{Z}$, where $\alpha = \sin^{-1}(1/\sqrt{5})$, or $x = \pi/2 + 2\pi k$, $k \in \mathbb{Z}$.
  \item Hint:
    \begin{itemize}
    \item Solve with Case IV.
    \item This problem was covered in class; please refer to your notes for a full solution.
    \end{itemize}
  \end{itemize}  


  \section*{Problem 6}

\begin{itemize}
  \item Problem statement: $\cos \theta + 1  = \sin \theta$, with $\theta \in [0, 2\pi]$.
  \item Solution: $\theta = \pi/2$ or $\theta = \pi$.
  \item Hint:
    \begin{itemize}
    \item Two possible approaches: either use Case IV to solve directly, or take square of both sides and expand (but be mindful of spurious solutions).
    \item This problem was covered in class; please refer to your notes for a full solution.

    \end{itemize}
  \end{itemize}  

  \section*{Problem 7}

\begin{itemize}
  \item Problem statement: $\cos \theta + \cos 2\theta + \cos 3 \theta = 0$.
  \item Solution: $\theta = \pm \pi/4 + \pi k$, $k \in \mathbb{Z}$, or $\theta = \pm 2\pi/3 + 2\pi k$, $k \in \mathbb{Z}$.
  \item Hint: Use addition formula to write $\cos \theta + \cos 3 \theta$ as $2 \cos 2\theta \cos \theta$ and factor out $\cos 2 \theta$.
    
  \end{itemize}  


\end{document}
