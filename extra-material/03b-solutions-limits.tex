\documentclass{amsart}

\usepackage[a4paper]{geometry}
\usepackage{parskip}

\title{Solutions for limit problems, slide deck 03b}

\begin{document}
\maketitle

\section*{Problem 1}

\begin{itemize}
\item Solution:
  \[
    \lim_{x \to +\infty} \frac{2x^5 + 1}{x^5 + x^3 + 1} = 2.
  \]
\item Hint: Look at the behavior of the leading-order terms ($x^5$ in the numerator and denominator). To solve rigorously, factor out $x^5$ from both numerator and denominator, and use limit rules.
\end{itemize}


\section*{Problem 2}

\begin{itemize}
\item Solution:
  \[
    \lim_{x \to -\infty} \frac{x^7}{\sqrt{x^{14} + 1}} = - 1.
  \]
\item Hint: Look at the behavior of the leading-order terms ($x^7$ in the numerator, $\sqrt{x^{14}} = - x^7$ in the denominator) and compare. To solve rigorously, factor out $x^7$ from the denominator (covered in class).
\end{itemize}


\section*{Problem 3}

\begin{itemize}
\item Solution:
  \[
    \lim_{x \to +\infty} \left(\sqrt{x^2 + 1} - x \right) = 0.
  \]
\item Hint: Multiply by the conjugate expression.
\end{itemize}


\section*{Problem 4}

\begin{itemize}
\item Solution:
  \[
    \lim_{x \to 0-} e^{1/x} = 0.
  \]
\item Hint: Introduce $y = 1/x$. As $x \to 0-$, $y \to -\infty$, so that $\displaystyle \lim_{x \to 0-} e^{1/x} = \lim_{y \to -\infty} e^y$.
\end{itemize}


\section*{Problem 5}

\begin{itemize}
\item Solution: This limit does not exist.
\item Hint: From the graph. The function $y = \sin x$ continues to oscillate between -1 and +1 as $x \to +\infty$.
\end{itemize}


\section*{Problem 6}

\begin{itemize}
\item Solution:
  \[
    \lim_{x \to \pi/2-} e^{\tan x} = +\infty.
  \]
\item Hint: Introduce $y = \tan x$. As $x \to \pi/2-$, $y \to \infty$, so that $\displaystyle \lim_{x \to \pi/2-} e^{\tan x} = \lim_{y \to \infty} e^y$.
\end{itemize}

\end{document}
